\chapter{Zusammenfassung}
\label{chapter:summary}
    Zielsetzung des Projekts war es, eine Plattform zu schaffen, mithilfe dessen das Empiriepraktikum des Instituts für Psychologie verwaltet werden kann.
    Hierbei sollte vor allem die Möglichkeit der automatischen Verteilung der Studenten anhand von Präferenzlisten umgesetzt werden.
    Diese und weitere Anforderungen wurden im Rahmen dieser Arbeit herausgearbeitet.
    Vor allem die Aufteilung in die verschiedenen Rollen Studenten, Dozenten und Administratoren mit unterschiedlichen Sichten auf das System waren ein wichtiger Anspruch.
    Basierend auf diesen Anforderungen wurde ein Entwurf für das System entwickelt und vorgestellt.
    Dieser umfasste neben Mockups für die wichtigsten Seiten der Plattform die Aufteilung in ein Front- und Backend mit dazugehörigem Datenmodell.
    Anschließend konnte die Umsetzung der Entwürfe erfolgen.
    Große Teile wurden dabei mit dem Framework \textit{October CMS} realisiert.
    Nachdem der Funktionsrahmen der Plattform geschaffen war, konnte der Verteilungsalgorithmus entwickelt werden.
    Eine geeignete lineare Zielfunktion wurde aufgestellt und eine mögliche Wahl der Parameter und weitere Varianten der Zielfunktion diskutiert.
    Dann konnte der Verteilungsalgorithmus in die bestehende Plattform eingefügt werden.\\
    
    Im Anschluss wurden die Tests der Plattform zur Einschreibung und Verwaltung und des Verteilungsalgorithmus betrachtet.
    Anhand der Tests konnte gezeigt werden, dass alle Kernfunktionen des Systems wie gewünscht funktionieren.
    Der Verteilungsalgorithmus erreichte auf künstlich generierten Datensätzen gute Verteilungen.
    Im Vergleich mit dem Algorithmus der alten Lösung, konnten ein deutlich besseres Ergebnis der Verteilung erzielt werden.\\
    
    Es konnten nicht alle Anforderungen umgesetzt werden.
    So wurde zum Beispiel auf die Implementierung einer Tauschbörse verzichtet.
    Ebenso wurde zwar das Backend zur Verwaltung verschiedener Module ausgelegt, nicht aber das Frontend.
    Diese beiden Erweiterungen sind eine Weiterführung der Arbeit an der Einschreibungsplattform naheliegend.\\
    
    \todo{noch irgend ein schlauer Schlusssatz der mit Zusammenfassend... beginnt}
    




    %\section{Stand des Projekts}
        
    %\section{Offene Probleme}
    
    %\section{Mögliche weitere Arbeiten}