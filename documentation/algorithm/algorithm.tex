\chapter{Verteilungsalgorithmus}
\label{chapter:algorithm}
    Im folgenden Kapitel wird der Algorithmus zum Verteilen der Studenten anhand der Präferenzlisten auf die Kurse vorgestellt.
    Zunächst wird im ersten Abschnitt eine geeignete Zielfunktion aufgestellt und kurz erklärt.
    Anschließend werden verschiedene Varianten für die Parameter diskutiert.
    \section{Zielfunktion}
%            Die Grundlegende Idee der Zielfunktion hat die Form:
%                $$ \max ~\text{Summe der Prioritäten} - \text{Gewicht} \cdot \text{Varianz} ~~~.$$
%            Genauer ausformuliert ergibt sich:
%                $$ \max 
%                    \sum_{i=1}^{n} \sum_{j=1}^{m} c(i,j)x_{ij} 
%                    - \frac{\beta}{n} \sum_{i=1}^{n}
%                        \left[\left(\sum_{j=1}^{m} c(i,j)x_{ij}\right) - \frac{1}{n} \sum_{i=1}^{n} \sum_{j=1}^{m} c(i,j)x_{ij}\right]^2 ~~~,$$
%            wobei gilt:\\
%                \begin{tabular}{l c l}
%                    $n$ & - & Anzahl der Studenten \\
%                    $m$ & - & Anzahl der Kurse\\
%                    $ c(i,j) $ & - & Priorität von Student $ i $ für Kurs $ j $\\
%                    $ \beta $ & - & Gewichtung der Varianz\\
%                    $t_{\min}(j)$ & - & Minimale Anzahl der Teilnehmer für Kurs $ j $\\
%                    $t_{\max}(j)$ & - & Maximale Anzahl der Teilnehmer für Kurs $ j $ ~~~.\\
%                \end{tabular}\\
%            
%            Zusätzlich sind drei Nebenbedingungen notwendig, um das Problem angemessen darzustellen.
%            Zum einen sollen die $ x_{ij} $ nur die Werte 0 oder 1 annehmen können:
%                $$ x_{ij} \in \{0,1\} ~~~.$$
%            Des Weiteren soll jeder Student nur einem Kurs zugeteilt werden:
%                $$ \forall {i \in \{1,..,n\}}: \sum_{j=1}^{m} x_{ij} = 1 ~~~.$$
%            Zuletzt ist die Teilnehmerzahl für die Kurse begrenzt:
%                $$ \forall {j \in \{1,..,m\}}: t_{\min}(j) \leq \sum_{i=1}^{n} x_{ij} \leq t_{\max}(j) ~~~.$$

        Ausgangspunkt für die Zielfunktion des Verteilungsalgorithmus ist eine Matrix $ (x_{ij})_{i=1,...,n;j=1,...,m} \in \{0,1\}^{n \times m}$, wobei $ n $ die Anzahl der Studenten und $ m $ die Anzahl der Kurse ist.
        Ein Eintrag in dieser Matrix kodiert, ob ein Student einem Kurs zugeordnet wurde, auf folgende Weise:
        wurde ein Student $ i $ dem Kurs $ j $ zugeteilt, dann ist $ x_{ij} = 1 $, wenn nicht, dann gilt $ x_{ij} = 0 $. 
        Diese $ x_{ij} $ sind die Variablen der Zielfunktion.
        Das bedeutet, es wird nach einer Belegung der Einträge $ x_{ij} $ der Matrix gesucht, so dass in jeder Zeile der Matrix immer genau ein Eintrag auf $ 1 $ gesetzt ist.
        Das ist gleich Bedeutend mit der Aussage, dass jeder Student genau einem Kurs zugeteilt wurde.
            $$ \max \sum_{i=1}^{n} \sum_{j=1}^{m} c(i,j)x_{ij}  ~~~,$$
        dabei bezeichnet $ c(i,j) $ eine Gewichtsfunktion. 
                
                
                
        \section{Parameter}