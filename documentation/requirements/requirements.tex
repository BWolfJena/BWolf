\chapter{Anforderungsanalyse}
\label{chapter:requirements}    

	In diesem Kapitel werden die Anforderungen an die Plattform dargestellt.
	Zunächst wird in Abschnitt \ref{sec:systemrequirements} behandelt, wie ein Praktikumsmodul abläuft und welche Erfordernisse sich daraus für die einzelnen Teilnehmergruppen ergeben.
	Darauf folgt eine Beschreibung der Aufgaben des Verteilungsalgorithmus.
	Zuletzt werden die technischen Anforderungen an das System beschrieben.
	
	\section{Anforderungen an das System}
		\label{sec:systemrequirements}
		Gemäß der allgemeinen Aufgabenstellung aus Abschnitt \ref{sec:task} existieren verschiedene Sichtweisen auf das Systems.
		Zum einen die Sicht der Verantwortlichen für Praktikum und Kurse, zum anderen die der Studenten.
		Erstere teilt sich wiederum in den Blickwinkel der Dozenten der Kurse und der übergeordneten Verantwortlichen für das Empiriepraktikum, im Weiteren Administratoren genannt, auf.
		Im Folgenden werden zunächst die Anforderungen für Studenten, Dozenten und Administratoren anhand des chronologischen Ablaufs eines Praktikumsmoduls dargestellt.
		Anschließend wird nochmals für die jeweiligen Sichten eine kurze Übersicht gegeben, sowie weitere detailliertere Anforderungen genannt.\\
		
		Die Anforderungen sind durch ein Gespräch mit den Verantwortlichen für das Empiriepraktikum entstanden. 
		Nicht alle Anforderungen sind zwingend für die Funktion des Systems.
		Einige sind optionale Funktionalitäten.
		Es ist zu beachten, dass aus zeitlichen Gründen nicht alle Anforderungen im Rahmen des Projektes umgesetzt werden konnten.
		Sie werden der Vollständigkeit halber dennoch aufgeführt.
		
		\subsection{Ablauf eines Praktikumsmoduls}
			Ein Praktikum beginnt, indem die Administratoren, nachdem sie sich in einer Oberfläche angemeldet haben, ein neues Praktikumsmodul erstellen.
			Zu diesem Praktikumsmodul gehören neben generellen Informationen wie Name und Semester auch die besonderen Angaben, ab wann Studenten ihre Präferenzliste erstellen können und zu welchem Zeitpunkt die automatische Verteilung vorgenommen werden soll.
			Im Anschluss können die Dozenten, nachdem auch sie sich in einer entsprechenden Oberfläche angemeldet haben, ihre Kurse zu dem aktuellen Praktikumsmodul hinzufügen.
			Dabei sollen Kurse Angaben über Titel, Dozent, Teilnehmerzahl, Ort, Zeit, Beschreibung des Kurses und eine Literaturliste besitzen.
			Zusätzlich soll ein Kurs Informationen über den Lehrstuhl, sowie die Finanzierung erhalten, die jedoch nicht für Studenten einsehbar sein soll.
			Nachdem alle Dozenten ihre Kurse eingetragen haben, sollen die Administratoren die aktuelle Kursübersicht online stellen können, sodass jeder die Kurse einsehen kann.
			Besteht Interesse, das Empiriepraktikum in diesem Semester zu absolvieren, so sollen sich die Studierenden registrieren können.
			Nach der Registrierung können die Studenten ihre Präferenzliste erstellen und speichern.
			Dabei soll die Präferenzliste auch jederzeit vom Studierenden noch verändert werden können.
			Nach jeder Änderung soll die Präferenzliste des Student gespeichert werden.
			Nach Ablauf der zuvor von den Administratoren festgelegten Frist können die Studenten anhand ihrer aktuellen Version der Präferenzliste auf die Kurse automatisch verteilt werden.
			Diese Verteilung sollte möglichst gut sein, d.h. in Fall dieses Projekts eine möglichst gleichmäßige Verteilung der Studenten mit vorzugsweise geringer Varianz.
			Nach dieser Verteilung soll eine Nachbearbeitungsphase folgen, in der die Administratoren das Ergebnis der Verteilung bei Bedarf manuell ändern können.
			Nachdem das Ergebnis der Verteilung fixiert wurde, sollen alle beteiligten Studenten, Dozenten und Administratoren über das Ergebnis der Verteilung mittels einer E-Mail-Benachrichtigung informiert werden.
			Ab diesem Moment sollen die Teilnehmer jedes Kurses für jeden angemeldeten Benutzer sichtbar sein.
			Anschließend an diese Nachbearbeitungsphase, könnte sich optional eine Tauschphase anschließen, in der die Studenten selbstständig über eine geeignete Oberfläche ihre Kurse tauschen können.
			
		\subsection{Verschiedene Sichten im Überblick}
		\label{Sichten}
		
			\subsubsection{Sicht der Administratoren}
				Administratoren haben die Möglichkeit ein neues Praktikum zu erstellen.
				Dabei müssen sie Angaben über Name, Semester, Frist für die Anmeldung und Beginn der automatischen Verteilung festlegen.
				Sie sollen aktiv die Kursübersicht für das aktuelle Praktikum veröffentlichen können.
				Des Weiteren können Administratoren den Verteilungsalgorithmus in der Nachbearbeitungsphase erneut mit anderen Parametern starten und die Verteilungsergebnisse auch manuell verändern.
				Zusätzlich sollen Administratoren die angemeldeten Dozenten verwalten können, um z.B. Dozenten, die keine Kurse mehr anbieten, zu entfernen.
				Auch sollen die Administratoren Kurse als \glqq frei bleiben\grqq~ markieren können, sowie die maximale und minimale Teilnehmerzahl für alle Kurse fließend einstellen können, falls nötig.
				Die Administratoren sollen für Praktika auch die Möglichkeit erhalten, Filterfunktionen anzuwenden und sich so z.B. die Kurse nach Lehrstühlen sortiert anzeigen zu lassen.
			
			\subsubsection{Sicht der Dozenten}
				Ein Dozent soll die Möglichkeit haben, Kurse für das aktuelle Praktikum zu erstellen und sich seine Kurse älterer Praktika nochmals anzusehen, um sie als Vorlage für neue Kurse zu nutzen. 
				Für einen Kurs müssen sie folgendes angeben: Name, Titel, Dozenten, Zeit und Raum, Teilnehmerzahl, eine Kurzbeschreibung des Kurses, eine ausführliche Beschreibung und optional eine Literaturliste.
				Diese Angaben sind auch für Studenten einsehbar.
				Dabei kann die Teilnehmerzahl nur auf fünf oder zehn gesetzt werden.
				Nicht für Studenten einsehbar sind die Angaben über Lehrstuhl, Lehrauftrag und die Information, ob der Dozent das erste Mal ein Empiriepraktikum leitet.
				
				Der Inhalt von längeren Angaben wie der Beschreibung des Praktikums, soll in einer Textumgebung möglich sein, in der geeignete Textformatierung möglich ist.
				Außerdem sollen auch Bilder in die Beschreibung eingearbeitet werden können.
				Des Weiteren sollen Dozenten nur ihre eigenen Kurse editieren können.
				Nachdem die Studenten verteilt worden sind, sollen die Dozenten eine E-Mail mit den Studenten, die in ihrem Kurs sind, erhalten.
				
				%                \todo{Im Verlaufe des Semesters können Dozenten ihren Kurs in zwei Kurse aufteilen.}
		
		
			\subsubsection{Sicht der Studenten}
				Die Studenten sollen ohne Registrierung die Kursübersicht aufrufen können, sobald die Administratoren die Kursübersicht veröffentlicht haben.
				Nach der Registrierung sollen die Studenten bis zu einer Frist eine Präferenzliste erstellen und späterhin auch bearbeiten können.
				Bei jeder angenommenen Änderung der Präferenzliste soll der Student über seine aktuelle Wahl per E-Mail informiert werden.
				Nachdem die Verteilung vom Algorithmus vorgenommen wurde, sollen die Studenten über ihr Ergebnis informiert werden, sowie die Ergebnisse der gesamten Verteilung einsehen können.
				Anschließend sollen sie eventuell die Möglichkeit erhalten, eigenständig die Kurse mit anderen Studenten zu tauschen, sofern beide zustimmen.
				
				
				%    	\todo{Das ganze offen und Modular halten für andere Module als das Empiriepraktikum}\\
				%    	\todo{Dabei auf Hierarchie von Präferenzen achten}\\
				%    	\todo{evtl. die Möglichkeit falls mehr Studenten als Kurse, neue Kurse hinzufügen können}\\
				%    	\todo{evtl. ein Archiv}\\
				%    
	
	
	
	\section{Der Verteilungsalgorithmus}
		Der Verteilungsalgorithmus verteilt alle Studenten auf die Kurse.
		Dabei ist es wichtig, dass die aufaddierte maximale Teilnehmeranzahl der Kurse größer ist als die Anzahl der zu verteilenden Studenten.
		Sollte dem nicht der Fall sein, so wird der Administrator informiert.
		Dieser hat nun die Möglichkeit die maximale Teilnehmeranzahl von Kursen zu erhöhen.
		Die Eingabe des Algorithmus sind die Gewichte der Präferenzen.
		
		Startet der Algorithmus, so versucht er, für jeden Studenten die größte, mögliche Präferenz zu den Kursen zu wählen.
		Dabei ist insbesondere wichtig, dass die Streuung der gewählten Präferenzen möglichst gering ist.
		Dies erfolgt beispielsweise durch eine Gewichtung der Präferenzen.\\
		Weiterhin platziert der Algorithmus in jeden Kurs mindestens drei Studenten, damit der Kurs sinnvoll angeboten werden kann.
		Es werden allerdings nie mehr Teilnehmer einem Kurs zugeordnet als die maximale Teilnehmeranzahl.
		
		Hat der Algorithmus schließlich eine passende Zuordnung von Studenten zu Kursen gefunden, so kann der Administrator die Ergebnisse bearbeiten, Parameter des Algorithmus neu einstellen, Kurse aktivieren/deaktivieren und ihn erneut starten.
		Ist der Administrator zufrieden mit der Verteilung, so bestätigt dieser das Ergebnis.
		Anschließend wird an die Dozenten eine E-Mail mit ihren Teilnehmern geschickt, und Studenten erhalten eine E-Mail mit ihrem Kurs.
		
	\section{Technische Details}
		
		Besucher der Website können sowohl auf mobilen Endgeräten, als auch per Desktop die Website besuchen.
		Diese passt sich dynamisch an die Bildschirmauflösung an.\newline
		Benutzt ein Besucher jedoch einen Browser, der älter als Internet Explorer 11 ist, so wird ihm mitgeteilt, er solle seinen Browser aktualisieren.\newline
		
		Der Login wird über eine E-Mail/Passwort Authentifizierung realisiert.
		Dafür sind nur E-Mail-Adressen der Friedrich-Schiller-Universität Jena nutzbar.
		Dies bedeutet, dass E-Mails auf \glqq @uni-jena.de\grqq~enden müssen.
		Die An- beziehungsweise Abmeldung ist hierbei zu jeder Zeit möglich.\newline
		
		Bei der Wahl der Präferenzen darf keine Präferenz doppelt belegt werden, d.h. jede Präferenz hat einen eindeutigen Kurs. \newline
		
		Der Verteilungsalgorithmus kann mit verschiedenen Optionen parametrisiert werden. Optionen können unter anderem die Gewichtung der Varianz, oder welche Kurse belegt werden können, sein. Weiterhin muss der Verteilungsalgorithmus innerhalb von 24 Stunden fertig sein. \newline
		
		Zu Zwecken der Portabilität wird das Produkt in einer docker-compose Umgebung ausgeführt.
		In dieser Umgebung soll die Website auf OctoberCMS mit Laravel 5.5 aufsetzen.
		Für den Webserver ist Nginx zu nutzen.
		Die Datenbank muss auf MySQL basieren.
		Zuletzt soll ein Cache-Server verwendet werden.
		Vorzugsweise ist dies Redis.\newline
		
		Die wesentlichsten Bestandteile der Website müssen über Tests validiert werden.
		Insbesondere der Verteilungsalgorithmus muss getestet werden.
		
