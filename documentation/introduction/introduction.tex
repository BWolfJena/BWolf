\chapter{Einleitung}
\label{chapter:introduction}

    \section{Motivation}
    \label{sec:motivation}
        Jedes Jahr (\textcolor{red}{Semester?}) wird an der Friedrich-Schiller-Universität Jena das Empiriepraktikum vom Institut für Psychologie angeboten.
        Dabei handelt es sich um eine Pflichtveranstaltung, die jeder Psychologie-Student absolviert haben muss.
        Innerhalb eines Praktikums werden mehrere Kurse angeboten.
        An einem dieser Kurse muss der Studierende teilnehmen, um das Modul erfolgreich abzuschließen.
        Allerdings ist die Teilnehmerzahl eines jeden Kurses begrenzt.
        So kann nicht jeder der über einhundert Studenten zu seinem Wunschkurs zugelassen werden.
        Die naheliegende Lösung ist die Zuweisung zu den Kursen nach Geschwindigkeit der Studenten bei der Anmeldung.
        Allerdings ist diese Verteilungsstrategie mit viel Stress und Streit verbunden.
        Die Studenten müssen anders auf die Kurse verteilt werden.
        Hierzu dient dem Institut für Psychologie eine Zuweisung zu den Kursen mittels Präferenzlisten.
        Jeder Studierende erstellt eine Liste mit Präferenzen, wie gerne er an welchen Kurs teilnehmen möchte.
        Es liegt auf der Hand, dass diese Listen nun nicht alle per Hand ausgewertet werden können, sondern automatisch erfolgen muss.
        Für diese Aufgabe existieren bereits Lösungen.
        So können mit dem Studienverwaltungstool der FSU Jena Friedolin Studenten bereits Präferenzlisten erstellt und ausgewertet werden.
        Bei der Verteilung mithilfe von Friedolin kommt es jedoch zu Problemen, da bei dem erstellen der Präferenzliste getrickst werden kann.
        Aus diesem Grund hat das Institut für Psychologie bereits eine eigene Plattform geschaffen, die das aufnehmen der Präferenzliste und das Verteilen auf die Kurse übernimmt.
        Hierbei ist aber die Verteilung auf die verschiedenen Kurse oft nicht zufriedenstellend und es muss manuell das Ergebniss angepasst werden.
        Das mit dieser Arbeit verbundene Projekt beschäftigt sich nun mit dem erstellen einer neuen Plattform für das Empiriepraktikum und der Frage, wie sich die Studenten besser auf die Kurse verteilen lassen.
        
    \section{Aufgabenstellung}
    \label{sec:task}
        Es gilt eine webbasierte Plattform für die Verwaltung des Empiriepraktikums der FSU Jena zu erstellen.
        Hauptaufgabe dieser Plattform ist, den verantwortliche Mitarbeitern des Instituts für Psychologie zu ermöglichen, für ein kommendes Semester ein neues Praktiumsmodul mit allen zugehörigen Kursen anlegen zu können.
        Des Weiteren sollen die Studenten die Möglichkeit erhalten eine Präferenzliste zu erstellen, welche der angebotenen Kurse sie am liebsten besuchen möchten.
        Nach Ablauf einer vorher festzulegenden Frist sollen die Studenten gemäß ihrer Präferenzliste dann automatisch bestmöglich auf die Kurse verteilt werden.
        Hierzu soll ein geeigneter Verteilungsalgorithmus entwickelt werden.


    \section{Aufbau der Arbeit}
    \label{sec:structure}
       Zunächst werden in dieser Arbeit die Anforderungen an das die Plattform, sowie den Verteilungsalgorithmus ausgehend von der Aufgabenstellung aus Abschnitt \ref{sec:task} näher spezifiziert.
       Anschließend wird der Entwurf für die Umsetzung der Anforderungen an das zu entwickelnde System dargestellt.