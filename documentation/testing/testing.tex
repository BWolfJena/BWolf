\chapter{Test des Systems}
\label{chapter:testing}

	Das Testen dieses Softwaresystem wurde in drei Aufgabenbereiche unterteilt.
	Zuerst wurde das Frontend durchgängig getestet.
	Anschließend wurde das Backend überprüft.
	Letztendlich wurde der Algorithmus mit seinen Auswirkungen auf Richtigkeit untersucht.
	
	\section{Frontend}
		Um das Frontend ordnungsgemäß zu testen, muss zuerst die Homepage getestet werden.
		In diesem Fall genügt es, den Header zu testen.
		Dies bedeutet, dass sowohl die Textfelder (''Start'', ''Kursübersicht'', ''Meine Präferenzliste'', ''Ergebnis sehen'') des Headers, als auch die Reihenfolge derer überprüft wird.
		Zusätzlich werden die Texte der Buttons ''Anmelden'' und ''Registrieren'' und die Reihenfolge dieser geprüft.
		Dies geschieht in dem Testfall \textit{HomepageTest}.\newline
		
		Der Testfall \textit{RegisterTest} verifiziert, dass ein Nutzer sich ordnungsgemäß registrieren kann und anschließend auch mit diesen Daten anmelden kann.
		Dafür klickt der automatisierte Nutzer auf den ''Registrieren''-Button der Homepage und registriert sich.
		Anschließend ruft der User die Login-Seite auf und meldet sich an.
		Ein erfolgreicher Login wird durch die Abwesenheit des ''Anmelden''-Buttons sicher gestellt.
		Zu Ende des Testfalls wird der neu registrierte Nutzer gelöscht.\newline
		
		Der darauffolgende Testfall ist \textit{LoginTest}.
		Dieser testet, dass ein bereits registrierter Benutzer sich anmelden kann.
		Dafür wird zuerst ein Nutzer in der Datenbank kreiert.
		Der automatische User öffnet die Login-Seite und gibt die Daten des kreierten Nutzers ein.
		Der erfolgreiche Login wird durch die Abwesenheit des ''Anmelden''-Buttons sicher gestellt.
		Zu Ende des Testfalls wird der kreierte Nutzer aus der Datenbank gelöscht.\newline
		
		Manuell getestet wurde die Kursübersicht.
		Um die Kursübersicht zu testen, müssen zuerst Kurse im Backend erstellt werden.
		Diese werden anschließend im Frontend unter dem Menüpunkt ''Kursübersicht'' überprüft.
		Hierbei ist es wichtig, dass in dem Einführungstext das korrekte Jahr steht.
		Anschließend wird die Vorschau für die Kurse kontrolliert.
		Dafür wird sichergestellt, dass der Kurstitel, die Kurzbeschreibung, Ort und Zeit und ein Ausschnitt der ausführlichen Beschreibung zu sehen sind.
		Weiterhin muss jede Vorschau einen ''Details''-Button beinhalten, der zu der kompletten Kursbeschreibung führt.
		Diese muss den Anforderungen entsprechen.\newline
		
		Anschließend wurde der Menüpunkt ''Meine Präferenzliste'' überprüft.
		Hier müssen alle Kurse angezeigt werden, die in diesem Semester gewählt werden können.
		Insbesondere ist es wichtig, dass jeder Kurs einzeln per Drag and Drop verschoben werden kann.
		Die verschobene Liste muss nun gespeichert werden können.
		Dies wird verifiziert, indem die Cookies und der Cache geleert werden und anschließend die Seite neu aufgerufen wird.
		Nun muss die Reihenfolge, in welcher die Kurse angeordnet wurden, mit der Reihenfolge der neu geladenen Seite übereinstimmen.\newline
		
		Der Menüpunkt 'Ergebnis sehen' wird überprüft, indem sicher gestellt wird, dass alle Nutzer, die eine Präferenzliste abgeschickt haben, verteilt wurden.\newline
		
	\section{Backend}
	
	\section{Algorithmus}
	
