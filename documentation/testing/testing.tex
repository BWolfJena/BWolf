\chapter{Test des Systems}
\label{chapter:testing}

	Das Testen dieses Softwaresystem wurde in drei Aufgabenbereiche unterteilt.
	Zuerst wurde das Frontend durchgängig getestet.
	Anschließend wurde das Backend überprüft.
	Letztendlich wurde der Algorithmus mit seinen Auswirkungen auf Richtigkeit untersucht.
	
	\section{Frontend}
		Um das Frontend ordnungsgemäß zu testen, muss zuerst die Homepage getestet werden.
		In diesem Fall genügt es, den Header zu testen.
		Dies bedeutet, dass sowohl die Textfelder (''Start'', ''Kursübersicht'', ''Meine Präferenzliste'', ''Ergebnis sehen'') des Headers, als auch die Reihenfolge derer überprüft wird.
		Zusätzlich werden die Texte der Buttons ''Anmelden'' und ''Registrieren'' und die Reihenfolge dieser geprüft.
		Dies geschieht in dem Testfall \textit{HomepageTest}.\newline
		
		Der Testfall \textit{RegisterTest} verifiziert, dass ein Nutzer sich ordnungsgemäß registrieren kann und anschließend auch mit diesen Daten anmelden kann.
		Dafür klickt der automatisierte Nutzer auf den ''Registrieren''-Button der Homepage und registriert sich.
		Anschließend ruft der User die Login-Seite auf und meldet sich an.
		Ein erfolgreicher Login wird durch die Abwesenheit des ''Anmelden''-Buttons sicher gestellt.
		Zu Ende des Testfalls wird der neu registrierte Nutzer gelöscht.
		
		Der darauffolgende Testfall ist \textit{LoginTest}.
		Dieser testet, dass ein bereits registrierter Benutzer sich anmelden kann.
		Dafür wird zuerst ein Nutzer in der Datenbank kreiert.
		Der automatische User öffnet die Login-Seite und gibt die Daten des kreierten Nutzers ein.
		Der erfolgreiche Login wird durch die Abwesenheit des ''Anmelden''-Buttons sicher gestellt.
		Zu Ende des Testfalls wird der kreierte Nutzer aus der Datenbank gelöscht.
					
	\section{Backend}
	
	\section{Algorithmus}
	
