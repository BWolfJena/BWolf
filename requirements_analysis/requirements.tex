\documentclass[12pt,a4paper]{article}

%\RequirePackage[margin=3.0cm,a4paper]{geometry}

\usepackage{times}
\usepackage[T1]{fontenc} 
\usepackage[utf8]{inputenc} 
\usepackage[ngerman]{babel}
\usepackage{graphicx, xcolor, hyperref} 
\usepackage{setspace} %mit /onehalfspacing wird im Textein 1,5 facher Zeilenabstand verwendet
\usepackage{accents}

\parindent 0pt


\begin{document}
    \onehalfspacing
    
    \begin{titlepage}
        \centering
        \includegraphics[width=0.4\textwidth]{Zweiter_Versuch.png}\par\vspace{1cm}
        {\scshape \LARGE BWolf \\ \Large Eine webbasierte Plattform zur
            Einschreibung und Verwaltung des
            Empiriepraktikums an der FSU Jena\par}
        \vspace{1.5cm}
        {\huge\bfseries Anforderungsanalyse\par}
        \vspace{1.5cm}
        {\large\itshape Christoph Keiner, Matthias Reuse, Ingo Schäfer, Christoph Staudt\par}
        \vspace{1.0cm}
        {\large \today\par}
    \end{titlepage}

    \pagenumbering{Roman}
    \tableofcontents 
    \clearpage
    \pagenumbering{arabic}
    
    \section{Allgemeine Problemstellung}
    \label{Problemstellung}
        Im Zuge des Projektes gilt es eine webbasierte Plattform zu schaffen, die die Verwaltung des Empiriepraktikums der FSU Jena ermöglicht.
        Hautaufgabe dieser Plattform ist, den verantwortliche Mitarbeitern der FSU Jena zu ermöglichen, für ein kommendes Semester ein neues Praktiumsmodul mit allen zugehörigen Kursen anlegen zu können.
        Des Weiteren sollen die Studenten die Möglichkeit erhalten eine Präferenzliste zu erstellen, welche der angebotenen Kurse sie am liebsten besuchen möchten.
        Nach Ablauf einer vorher festzulegenden Frist sollen die Studenten gemäß ihrer Präferenzliste dann automatisch bestmöglich auf die Kurse verteilt werden.     
    
    \section{Genaue Anforderungen an das System}
        Gemäß der allgemeinen Problemstellung aus Abschnitt \ref{Problemstellung} existieren verschiedene Sichtweisen auf die Anforderungen des Systems.
        Zum einen die Sicht der Verantwortlichen für Praktikum und Kurse, zum anderen die der Studenten.
        Erstere teilt sich wiederum auf in den Blickwinkel der Dozenten der einzelnen Kurse und der übergeordneten Verantwortlichen für das Empiriepraktikum, im Weiteren Administratoren genannt.
        Im Folgenden werden zunächst die Anforderungen für Studenten, Dozenten und Administratoren anhand des chronologischen Ablaufs eines Praktikumsmoduls dargestellt.
        Anschließend wird in Abschnitt \ref{Sichten} nochmal eine kurze Übersicht über die Anforderungen der jeweiligen Sichtweisen gegeben.
        
        \subsection{Ablauf eines Praktikumsmoduls}
            Der Ablauf beginnt, indem die Administratoren, nachdem sie sich in einer Login-Oberfläche angemeldet haben, ein neues Praktikumsmodul erstellen.
            Zu diesem Praktikumsmodul gehören neben generellen Informationen wie Name und Semester auch die besonderen Angaben, ab wann Studenten ihre Präferenzliste erstellen können und zu welchem Zeitpunkt die automatische Verteilung vorgenommen werden soll.
            Im Anschluss können die Dozenten, nachdem auch sie sich in einer entsprechenden Oberfläche angemeldet haben, ihre Kurse zu dem aktuellen Praktikumsmodul hinzufügen.
            Dabei sollen Kurse Angaben über Titel, Dozent, Teilnehmerzahl, Ort, Zeit, Beschreibung des Kurses und Literaturliste besitzen.
                    
        \subsection{Verschiedene Sichten im Überblick}
        \label{Sichten}
                    
            \subsubsection{Sicht der Administratoren}
            Ein Administrator hat die Möglichkeit, neue Dozenten zu erstellen, diese zu bearbeiten und zu löschen.
            Zusätzlich ist er dafür verantwortlich, die Ergebnisse des Verteilungsalgorithmus zu verändern und zu bestätigen.
                 
                
            \subsubsection{Sicht der Dozenten}
            Ein Dozent hat die Möglichkeit, Kurse für das Praktikum zu erstellen. Für einen Kurs müssen sie folgendes angeben:
            \begin{itemize}
            \item Name
            \item Titel
            \item Dozenten
            \item Zeit/Raum
            % Minimale Teilnehmer fällt raus, da die Psychologen festegelegt haben, es müssen mindestens 3 Leute in jedem Kurs sein
            \item Maximale Teilnehmer (5 oder 10!)
            \item Kurzbeschreibung
            \item Beschreibung
            \item Literatur
            \item E-Mail Adresse des Empirie-Praktikums-Leiter %? Ist das nicht nur die vom Dozenten
            \end{itemize}
            Folgendes ist nicht für den Studenten einsehbar:
            \begin{itemize}
            \item Lehrstuhl
            \item Lehrauftrag
            \item Erstes Mal Empiriepraktikum?
            \end{itemize}
            
            Nachdem die Studenten verteilt worden sind, erhalten die Dozenten eine E-Mail mit den Studenten, die in ihrem Kurs sind.
            Im Verlaufe des Semesters können Dozenten ihren Kurs in zwei Kurse aufteilen.
            
            Sollten Dozenten zu einem anderem Semester erneut Kurse anbieten, so haben sie die Möglichkeit, ihre alten Kurse einzusehen und (ohne Änderung des alten Kurses) einen dieser als Vorlage für einen neuen Kurs zu nutzen.

                
            \subsubsection{Sicht der Studenten}
            %Studenten erhalten unabhängig dieses Systems eine Woche vor Beginn der Einschreibung einen Link (Dies muss nicht abgebildet werden, deswegen nehmen wir das nicht auf, oder?)
            3 Wochen vor Verteilung der Kurse beginnt die Einschreibungszeit. Hierfür erhalten die Studenten unabhängig von diesem System eine Benachrichtigung.
            Studenten sehen bei ihrem Besuch zuerst eine kurze Erläuterung zu dem Einschreibungs-System. Neben dem Text gibt es noch ein Feld zur Anmeldung bzw. Registrierung.
            Nachdem ein Student sich registriert hat und angemeldet ist, wird er zur Liste der derzeitigen Kurse weitergeleitet. Diese Kurse können allerdings auch ohne Anmeldung angesehen werden. % Alles soll öffentlich sein, aber gleichzeitig haben wir das so formuliert, dass der Student dann zu den Kursen kann. Weiß aber nicht, wie sinnvoll die Weiterleitung zu den Kursen ist.
            
            Durch eine weitere Seite, können Studenten dann jeden Kurs nach ihrer Präferenz sortieren. Dabei müssen Studenten jedem Kurs eine Präferenz zuteilen. Sollten Studenten dann ihre Präferenzenliste bestätigen, so wird ihnen an ihre hinterlegte E-Mail-Adresse ihre gewählte Präferenzenliste geschickt. Diese Liste kann beliebig verändert werden, bis die Verteilung beginnt. %Die Psychologen meinten auch sowas, dass sie das bestätigen müssen und dann ist es fest, aber sie können es noch ändern wie sie wollen. Weiß jemand genaueres?
            Sobald die Verteilung abgeschlossen und durch einen Administrator bestätigt wurde, erhalten Studenten eine E-Mail-Benachrichtigung, in dem ihr zugeteilter Kurs drin steht.
            Studenten sehen nun auch für alle Kurse die Namen der Teilnehmer.
            
            Nach der Verteilung beginnt die Tauschperiode. Bis zum Beginn der ersten Vorlesungswoche können Studenten Tauschanfragen an andere Studenten schicken. Jede dieser Tauschanfragen wird automatisch an den Empirie-Praktikums-Leiter weitergeleitet.
            Ist eine Tauschanfrage erfolgreich, so tauschen die Studenten ihre Kurse.
    
    \section{Der Verteilungsalgorithmus}
    Der Verteilungsalgortihmus muss folgenden Anforderungen gerecht werden:
    \begin{itemize}
%    \item Nach Ende der Einschreibung, startet der Algortihmus automatisch. -> unsicher ob sie es so wollten durch die verschiedenen Einstell-Möglichkeiten. Sie hatten im Gespräch beides gesagt.
	\item Er verteilt Studenten auf die Kurse.
    \item Er versucht die maximale Punktzahl anhand der Präferenzlisten zu erreichen.
    \item Er überschreitet nicht die maximale Kursteilnehmer-Anzahl mit seiner Verteilung.
    \item Die Streuung der Präferenzen der Studenten ist gering.
    \item Kurse können für den Algorithmus deaktiviert werden. Dies bedeutet, dass in diesen Kurs niemand zugeteilt wird.
    \item Jedem aktivierten Kurs werden mindestens 3 Leute zugeteilt.
    \item Die Verteilung ist wirksam, wenn sie von einem Administrator bestätigt wird.
    \item Die Verteilung hat verschiedene Parameter:
    	\begin{itemize}
    	\item Gewichte der Präferenzen
    	\item Auswahl des Optimierungsalgorithmus %ist optional afaik
    	\end{itemize}
    \item Sollte eine Zuteilung aller Studenten nicht möglich sein, so wird der Administrator benachrichtigt.
    \end{itemize}
    
    \section{Technische Details}
    
    
\end{document}


%Allgemeine Produkt Anforderungen
%User Flow
%Student
%
%Start-Screen mit kurzem Erklärungstext und Anmeldung
%Oauth / Shiboleth URZ, aber unklar ob und wie wir das benutzen können
%Einfaches Passwort Login, dass nur uni-emails akzeptiert um Fehler zu vermeiden.
%Allgemeine Informationen und spezifische zu den Kursen
%Welche Informationen brauchen wir über Kurse?
%Name, Titel, Dozenten, Zeit/Raum, Minimale Teilnehmer, Maximale Teilnehmer (5 oder 10!), Kurzbeschreibung, Beschreibung, Literatur.
%Wahl der Prioritäten, für alle? Kurse? Evtl. egal priorisierung bei der bei Verteilung dann ein belieber Kurs gewählt wird der für die Allgmeinheit am besten ist
%Jede Priorität maximal einmal
%Wahl sollte geändert werden können bis die Einschreibungsperiode endet
%Bei jeder Änderungen sollte eine E-Mail mit der aktuellen Liste geschickt werden
%Algorithmus zur Verteilung wird zum Ende der Einschreibungsperiode getriggert
%Minimierung der Varianz
%Implementierung meherere Algorithmen? 4.5 Erste manuelle Nachbearbeitung und dann Notification / Veröffentlichung
%Anzeigen verschiedener Parameter, ändern der Parameter.
%Benachrichtigung über Ergebnis der Studenten via E-Mail
%Optional: Auch eine eigene Seite auf der Webseite mit dem Ergebniskurs? Anzeige anderer Teilnehmer?
%Studenten sollten alle sehen können
%Optional Tauschsystem (Studenten können gezielt andere Studenten anfragen und annehmen und akzeptieren)?
%Vor der ersten Vorlesungswoche automatisiert mit CC an den Empraleiter und die Superadmins.
%Beenden mit Beginn der ersten Vorlesungswoche
%Login / Logout beliebig
%
%Verantwortlicher
%
%Erstellen eines neuen Praktikums
%
%Name, Titel, Beschreibung?, Einschreibungsperiode, Beginn?, Ende?
%
%Erstellen von Kursen für das Praktikum
%Kursverantwortlicher erhält E-Mail mit Teilnehmern? --> Ja!
%Nachbearbeitung des Ergebnis des Algorithmus
%Auch über die maximale Teilnehmeranzahl hinaus, eher nicht!
%Bearbeiten und Löschen von Kursen und Praktikas
%
%Superadmin
%
%Erstellen, Bearbeiten und Löschen von Verwantworlichen / Betreuen
%
%Verteilungsalgorithmus
%
%Piroritätenliste von 1-Anzahl Kurse (+ egal option?), es gibt ausreichend Plätze
%Wenn es mehr gibt werden Leute vorher (zufällig) rausgeschmissen, evtl Liste mit Usern die hier nicht gekickt werden könne
%Zumindest Fehlermeldung ausgeben!
%Benachrichtigung wenn es zu viele werden?
%Maximal 24h Lauftzeit
%Was ist das Optimum?
%
%Endgeräte
%
%Mobil unterstützüng
%Welche Desktopbrowser? Reicht IE11 (neuster auf Win7), dann können bessere Frameworks verwendet werden, wie Bootstrap 4 / Semantic UI
%Für ältere wird Hinweis angezeigt das sie ihren Browser updaten sollten
%Ansonsten Bootstrap 3 mit IE8 Unterstützung
%
%Software
%
%Idealerweise docker-compose Umgebung die wir selbst konfigurieren können
%Erste Wahl wäre OctoberCMS mit Laravel 5.5, das setzt voraus:
%PHP version 7.0 or higher
%PDO PHP Extension
%cURL PHP Extension
%OpenSSL PHP Extension
%Mbstring PHP Library
%ZipArchive PHP Library
%GD PHP Library
%Webserver, am liebsten Nginx, alternativ auch Apache
%Notfalls: Java mit dem Playframework, dies benötigt Java 1.8
%Datenbank, PostgreSQL, MariaDB oder MySQL
%Cache-Server: Redis o. ähnliches
%Tests für die wesentlichsten Bestandteile (insbesondere des Algorithmus)
%
%Offene Fragen
%
%Gibt es mehrere Praktika mit mehreren mit gleicher/ähnlicher Logik? Oder immer nur eins und es wird immer nur das neuste angezeigt?
%Einschreibung in mehrere Kurse? (über die Jahre hinweg / durchgefallen?)
%Werden bestimmte Kurse immer wieder ähnlich angeboten?
%Sollen für Teilnehmner kursbezogene Informationen / Dateien angeboten werden? -- Nein!
%Sollen die Daten über einen längeren Zeitraum gespeichert werden? -- ja macht sinn für Dozenten
%Sollen die weitere Veranstalttungsinformationen öffentlich einsehbar sein? JA! - Evtl auch die Kurse im letzten Jahr anzeigen
%Welche Informationen brauchen wir über Kurse? - s.o. - Lehrstühle verwalten - Bei der übersicht der Kurse Lehrstuhl anzeigen und filtern können. - Filter für Teilnehmer - Finanzierung: Haushaltsmittel vorhanden? - Ersters Seminar das vom Dozenten angeboten wird?
%Verteilung von Berechtigungen
%Erstellen, Bearbeiten und Löschen von Praktikas (Soll jeder alle oder nur seine eigenen bearbeien können?) --> Nein
%Erstellen, Bearbeiten und Löschen von Kurse für ein Praktikum (Auch hier jeder alle, nur bestimmte Praktikas?...) --> nur bestimmte
%Was passiert, wenn es weniger Plätze gibt als Anmeldungen? - Der Fall ist noch nicht vorgekommen. - Aber umgekehrt sollten wenn zu wenige da sind, sollte für jeden Kurs mindestens drei Teilnehmner haben. Idealerweise minimale Teilnehmeranzahl pro Kurs angeben könenn.
%Sollen Studenten vor der Verteilung in bestimmte Kurse festgesetzt werden können, bzw. im Nachhinein gelöscht / verschoben / hinzufgefügt werden können? --> Ja, am ende sollte man verschieben können
%Tauschsystem für Studenten? --> Wäre nett s.o.
%Anzeige anderer Studenten in dem Kurs? --> Ja
%Mitteilung and die Verantwortlichen wer im Kurs ist? -> Ja
%Was ist das Optimum für die Verteilung?
%
%Besprechung mit den Psyschologen
%
%Verteilung von Seminaren, typischer mehr Plätze als Studierende. Bei den Lehrstühlen wird angefragt, wie viele und welche Seminare Angeboten werden und wie viele Studenten in das Seminar passen. Präferenzliste wird erst einige Wochen später angezeigt. Ziel bisher, möglichst vielen Studierenden ihrem maximal Kurs zuzuordnen, es wäre aber besser wenn es halt halbwegs gute Kurse bekommen. Wir brauchen mindestens 2-level admins, eine die nur Kurse erstellen können und eigene bearbeiten können.