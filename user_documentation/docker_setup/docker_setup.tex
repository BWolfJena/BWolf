\chapter{Installation der Docker-Umgebung auf Ubuntu}
\label{chapter:docker_setup}
    Im folgenden Kapitel wird die Installation der Docker-Umgebung auf Ubuntu mit einer amd64 Architektur erläutert.
    Für weitere Systeme werden in den Kapitel die nötigen Ressourcen verlinkt, aber nicht näher beschrieben.
    Zunächst wird die Installation der Docker Community Edition (Docker-CE) erläutert.
    Schließlich muss docker-compose installiert werden.
    Zuletzt wird erklärt, wie der Server gestartet werden kann.
        
    \section{Installieren der Docker Community Edition}
    	\subsection{Vorbereitung}
	        Zur Installation der Docker-CE gibt es bereits ein von Docker bereitgestellte Dokumentation\footnote{\href{https://docs.docker.com/install/linux/docker-ce/ubuntu/}{https://docs.docker.com/install/linux/docker-ce/ubuntu/}}.
    	    Aufgrunddessen wird hier nur die Konsolenbefehle aufgelistet, die verwendet werden müssen.
        	Entferne alte Docker Versionen mithilfe von:
	        \begin{lstlisting}
$ sudo apt-get remove docker docker-engine docker.io
			\end{lstlisting}
			Updaten Sie nun apt-get via:
			\begin{lstlisting}
$ sudo apt-get update
			\end{lstlisting}
			Laden Sie nun die Pakete herunter, die den Download eines Repositories via HTTPS ermöglichen:
			\begin{lstlisting}
$ sudo apt-get install \
    apt-transport-https \
    ca-certificates \
    curl \
    software-properties-common
			\end{lstlisting}
			Fügen Sie nun Dockers GPG Schlüssel hinzu
			\begin{lstlisting}
$ curl -fsSL https://download.docker.com/linux/ubuntu/gpg \
| sudo apt-key add -
			\end{lstlisting}
			Sie können nun den Schlüssel verifizieren, indem sie die Ausgaben vergleichen
			\begin{lstlisting}
$ sudo apt-key fingerprint 0EBFCD88
pub   4096R/0EBFCD88 2017-02-22
      Key fingerprint = 9DC8 5822 9FC7 DD38 854A \
                        E2D8 8D81 803C 0EBF CD88
uid                  Docker Release (CE deb) <docker@docker.com>
sub   4096R/F273FCD8 2017-02-22
			\end{lstlisting}
			Letztendlich fügen Sie hiermit das Docker Repository zu apt-get hinzu
			\begin{lstlisting}
$ sudo add-apt-repository \
   "deb [arch=amd64] https://download.docker.com/linux/ubuntu \
   $(lsb_release -cs) \
   stable"
			\end{lstlisting}
			
			\subsection{Installation}
			Updaten Sie nun erneut apt-get:
			\begin{lstlisting}
$ sudo apt-get update
			\end{lstlisting}
			Installieren Sie nun die neuste Version der Docker-CE:
			\begin{lstlisting}
$ sudo apt-get install docker-ce
			\end{lstlisting}
			Verifizieren Sie nun, dass Docker-CE installiert ist, indem sie den hello-world Container starten:
			\begin{lstlisting}
$ sudo docker run hello-world
			\end{lstlisting}
			Sollte keine Fehlermeldung kommen, sondern ein Bild heruntergeladen werden und Text angezeigt werden, so hat die Installation funktioniert.
        
    \section{Installieren von docker-compose}
        Zur Installation von docker-compose gibt es ebenfalls eine von Docker bereitgestellte Anleitung\footnote{\href{https://docs.docker.com/compose/install/}{https://docs.docker.com/compose/install/}}.
        Laden Sie zunächst docker-compose runter:
   	    \begin{lstlisting}
$ sudo curl -L https://github.com/docker/compose/\
releases/download/1.18.0/docker-compose-`uname -s`-`uname -m` \
-o /usr/local/bin/docker-compose
	    \end{lstlisting}
        Geben Sie nun docker-compose die benötigten Rechte, um es ausführen zu können:
   	    \begin{lstlisting}
$ sudo chmod +x /usr/local/bin/docker-compose
	    \end{lstlisting}
	    Verifizieren Sie nun, dass docker-compose installiert wurde, indem sie folgenden Befehl ausführen:
	    \begin{lstlisting}
$ docker-compose --version
	    \end{lstlisting}
        
    \section{Starten des Server}
        Der Server kann nun mit dem beigefügten Start-Skript gestartet werden.
        Dieses befindet sich im Root-Verzeichnis des Projekts.
        Sei der Pfas zum Root-Verzeichnis <BWolf>.
        Führen sie dafür folgenden Befehl aus:
   	    \begin{lstlisting}
$ ./<BWolf>/start.sh
	    \end{lstlisting}
	    Bei der ersten Ausführung des Befehls werden alle benötigten Dateien heruntergeladen.
	    Dies nimmt längere Zeit in Anspruch.
        Sofern keine Änderungen an der Docker-Umgebung gemacht werden, sollte jede weitere Ausführung des Befehls nur noch den Server starten.
        Der Server ist schließlich erreichbar über \href{http://localhost}{localhost}.