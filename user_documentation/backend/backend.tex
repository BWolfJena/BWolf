\chapter{Verteilungsalgorithmus}
\label{chapter:algorithm}
    Im folgenden Kapitel wird der Algorithmus zum Verteilen der Studenten anhand der Präferenzlisten auf die Kurse vorgestellt.
    Zunächst wird im ersten Abschnitt eine geeignete Zielfunktion aufgestellt und kurz erklärt.
    Anschließend werden verschiedene Varianten für die Parameter diskutiert.
        
    \section{Erweiterung der Zielfunktion und Wahl der Parameter}
        Die Parameter der Zielfunktion können frei gewählt werden.
        In diesem Abschnitt sollen verschiedene Varianten dargestellt werden, wie die Gewichte gewählt bzw. wie die Zielfunktion weiter angepasst und erweitert werden kann.
        
        \subsection{Naheliegende Wahl der Parameter}
            Die nahe liegende Wahl der Gewichte ist, die Präferenzliste der einzelnen Studenten für die Kurse zu verwenden.
            Da versucht wird, die $ x_{ij} $ so zu wählen, dass die Summe maximal wird, wird so eine Zuordnung eines Studenten zu einem Kurs, für den er eine höhere Präferenz angegeben hat, gegenüber einem mit einer niedrigen Präferenz bevorzugt.
            Der sogenannte \textit{Score} der Funktion, also das Ergebnis der Summe, wird somit höher, je mehr Studenten in von ihnen gewünschte Kurse verteilt werden.
        